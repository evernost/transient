\documentclass[10pt]{article}
\usepackage[utf8]{inputenc}
\usepackage[T1]{fontenc}
\usepackage{mathtools}
\usepackage{amssymb}
\usepackage{circuitikz}
\usepackage{fontspec}
\usepackage{array}
\usepackage{float}

\setmainfont{Times New Roman}

%
% NOTE: schematics generated using the online tool: 'https://www.circuit2tikz.tf.fau.de/designer'.
% Special thank to them!
%

\title{Voltage-Controller Filter}
\author{QuBi}
\date{\today}

\begin{document}

\maketitle

This document summarises all the design choices and technical justifications for the Voltage-Controlled Filter module.\\

The filter is based on a State Variable Filter.
Reasons:
\begin{itemize}
	\item versatile functions with a single block
	\item no drop in the bass at high Q
	\item possibility extend it to very different topologies (Chebychev, Elliptic)
\end{itemize}


\section*{Filter structure}

It follows that the cutoff frequency is directly proportional to the voltage gain.


\section*{Integrator Function}


\section*{Voltage Controlled Attenuator}

From the "Filter Structure" section, there is a linear relationship between the voltage gain and the cutoff frequency. Therefore, for a frequency sweep from 10 to 10kHz, the attenuation must reach a similar ratio i.e. at least 1:1000. Unlike a VCA, it is not necessary to reach a complete null signal.

Also, in order to get a \textit{spicy} resonance (high Q value), the attenuator must be capable of handling large signals (up to a few volts)

\section*{Annex 1: useful formulas}

\subsection*{Resistive Voltage Divider with target equivalent resistor}
With a resistor divider, there are multiple values for the 2 resistor to get the target ratio. 
However, there is only one pair of resistors $(R_1, R_2)$ that will have a given series resistor in the Thevenin equivalent.

\begin{circuitikz}
	\draw (10, 9.75) to[american voltage source, l={$\alpha V$}] (10, 7.75);
	\draw (10.25, 10) to[american resistor, /tikz/circuitikz/bipoles/length=0.700cm, l={$R_{TH}$}] (12.25, 10);
	\draw (10, 9.75) -- (10, 10) -- (10.75, 10);
	\draw (10, 7.75) -- (10, 7.5) -- (12.25, 7.5);
	\node[ocirc] at (12.25, 10){};
	\node[ocirc] at (12.25, 7.5){};
\end{circuitikz}

\begin{circuitikz}
	\draw (10, 9.75) to[american voltage source, l={$V$}] (10, 7.75);
	\draw (10.25, 10) to[american resistor, /tikz/circuitikz/bipoles/length=0.700cm, l={$R_{1}$}] (12, 10);
	\draw (10, 9.75) -- (10, 10) -- (10.75, 10);
	\draw (10, 7.75) -- (10, 7.5) -| (13, 7.5);
	\node[ocirc] at (13, 7.5){};
	\draw (12, 10) to[american resistor, /tikz/circuitikz/bipoles/length=0.700cm, l={$R_{2}$}] (12, 7.5);
	\draw (12, 10) -- (13, 10);
	\node[ocirc] at (13, 10){};
	\node[circ] at (12, 10){};
	\node[circ] at (12, 7.5){};
	\node[shape=rectangle, minimum width=1.465cm, minimum height=2.465cm] at (14.25, 8.75){} node[anchor=west, align=left, text width=1.077cm, inner sep=6pt] at (13.5, 8.75){$\alpha V$};
	\draw[line width=1pt, -latex] (13.5, 7.75) -- (13.5, 9.75);
\end{circuitikz}

This is the formula you need: 
$$
\left\{
	\begin{array}{ll}
		R_1 = \dfrac{R_{TH}}{\alpha} \\
    R_2 = \dfrac{R_{TH}}{1-\alpha}
	\end{array}
\right.
$$\\

\subsection*{Generalized resistor divider}


\section*{Annex 2: impact of non-linearities}
In this circuit, the main source of nonlinearities is the VCA.



\end{document}
