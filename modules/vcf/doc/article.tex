\documentclass[10pt]{article}
\usepackage[utf8]{inputenc}
\usepackage[T1]{fontenc}
\usepackage{mathtools}
\usepackage{amssymb}
\usepackage{circuitikz}
\usepackage{fontspec}
\usepackage{array}
\usepackage{float}

\setmainfont{Times New Roman}

%
% NOTE: schematics generated using the online tool: 'https://www.circuit2tikz.tf.fau.de/designer'.
% Special thank to them!
%

\title{Voltage-Controller Filter}
\author{QuBi}
\date{\today}

\begin{document}

\maketitle

This document summarises all the design choices and technical justifications.\\

Implementation is a State Variable Filter.
Reasons:
\begin{itemize}
	\item versatile functions
	\item no drop in the bass at high Q
	\item possibility extend it to very different topologies
\end{itemize}


\section*{Filter structure}

It follows that the cutoff frequency is directly proportional to the voltage gain.


\section*{Integrator Function}


\section*{Voltage Controlled Attenuator}

From the "Filter Structure" section, it follows that
In order to sustain very high resonance, the attenuator must endure

\begin{circuitikz}
	\node[eground2] at (2, 1){};
	\draw (2, 2.5) to[empty diode, l={$D$}] (2, 1);
	\draw (2, 4.5) to[american resistor, /tikz/circuitikz/bipoles/length=1.05cm, l={$R$}] (2, 2.5);
	\draw (2, 4.5) |- (1, 5);
	\node[ocirc](N1) at (1, 5){} node[anchor=east] at (N1.west){$V$};
	\draw[line width=1.6pt, -latex] (3.5, 1.25) -- (3.5, 2.5);
	\draw[dash pattern={on 0.4pt off 1.6pt}] (2.25, 2.75) -- (3.5, 2.75);
	\draw[dash pattern={on 0.4pt off 1.6pt}] (2.25, 1) -- (3.5, 1);
	\node[shape=rectangle, minimum width=0.715cm, minimum height=1.215cm] at (3.875, 1.875){} node[anchor=center, align=center, text width=0.327cm, inner sep=6pt] at (3.875, 1.875){$V_D$};
	\draw[line width=1.6pt, -latex] (2, 4.5) -- (2, 4.25);
	\node[shape=rectangle, minimum width=0.715cm, minimum height=1.215cm] at (2.5, 4.375){} node[anchor=center, align=center, text width=0.327cm, inner sep=6pt] at (2.5, 4.375){$I_D$};
\end{circuitikz}

Even though it is probably the simplest circuit involving a diode, the exponential behaviour of the $I/V$ curve takes a bit of method to solve the circuit.

Here are some common approaches.

\subsubsection*{Method 1: the constant diode voltage}
This is the most common approach when you want to adjust $R$ so that a current $I_D$ flows into the diode. 

Assume the diode voltage is constant ($V_D = 0.7V$ for silicon diodes or up to a few Volts for LEDs) \textit {i.e.} $V_D$ is independent of the current $I_D$. Then:
$$
	R = \dfrac{V - V_D}{I_D}
$$

Treating $V_D$ as a constant is a very common assumption and yields results that are precise enough for most applications. You find these in:
\begin{itemize}
	\item the 0.7V drop in the base-emitter junction in silicon BJTs
	\item diode voltage dropper	
	\item zener diode regulator
\end{itemize} 

It works pretty well as soon as enough current (even a few mA) starts flowing in the diode. In the \textbf{method 4} we will investigate some more the justification behind this.

\subsubsection*{Method 2: the iterative solving}
	\begin{enumerate}
		\item Propose a first \textit{guess} voltage accross the diode: $V_D = 0.7V$ (no need to be accurate)
		\item Determine the current $I_D$ using the diode $I/V$ model: $I_D = I_S \left( \exp \left( \dfrac{V_D}{V_T}\right)-1\right)$
		\item Calculate the new $V_D$: $V_D = V - R I_D$ 
		\item Go back to 2. until $V_D$ and $I_D$ eventually settle. 
	\end{enumerate}
	Convergence rate is \textit{crazy} fast and the result gets accurate to a few digits after only 2 or 3 iteration. 
	However, this method is more of a theoretical fact. It might be useful in a simulator, but for practical situations it won't bring much more precision. The natural dispersion in $I/V$ curves within the same batch of diodes, the lack of detailed information about the model (\textit{e.g.} $I_S$), not even mentioning the limits of the model itself lowers the benefit of this method.

\subsubsection*{Method 3: Analytical solving (Lambert's W function)}
Assume we know how to solve for $x$ in the equation $\exp(x) = ax + b$ with $a \in \mathbb{R}^*$ and $b \in \mathbb{R}$. 

Although the analytical method could in principle yield better results, the conclusions here are similar to the previous method: the solution can't be better than the weakest element of the model.

% TODO: precision analysis on V_D vs I_S,\eta,°T to show that they are huge contributors.
% Therefore, considering their dispersion, its pretty useless.






\end{document}
